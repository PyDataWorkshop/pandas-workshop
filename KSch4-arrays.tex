\section*{Array}
Arrays are the base data type in NumPy, are are arrays in some ways similar to lists since they both contain
collections of elements. The focus of this section is on homogeneous arrays containing numeric data
– that is, an array where all elements have the same numeric type (heterogeneous arrays are covered in
Chapters 16 and 17). 

Additionally, arrays, unlike lists, are always rectangular so that all rows have the same
number of elements.

Arrays are initialized from lists (or tuples) using array. Two-dimensional arrays are initialized using
lists of lists (or tuples of tuples, or lists of tuples, etc.), and higher dimensional arrays can be initialized by
further nesting lists or tuples.

%=======================================%
Matrix
Matrices are essentially a subset of arrays, and behave in a virtually identical manner. The two important
differences are:
• Matrices always have 2 dimensions
• Matrices follow the rules of linear algebra for *


%=======================================%

Accessing Elements of an Array
Four methods are available for accessing elements contained within an array: scalar selection, slicing,
numerical indexing and logical (or Boolean) indexing. Scalar selection and slicing are the simplest and so
are presented first. Numerical indexing and logical indexing both depends on specialized functions and
so these methods are discussed in Chapter 12
