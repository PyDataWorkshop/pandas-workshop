
Shape Information and Transformation
%-----------------------------------%
shape

shape returns the size of all dimensions or an array or matrix as a tuple. shape can be called as a function
or an attribute. shape can also be used to reshape an array by entering a tuple of sizes. Additionally, the
new shape can contain -1 which indicates to expand along this dimension to satisfy the constraint that
the number of elements cannot change.
>>> x = randn(4,3)
>>> x.shape
(4L, 3L)
>>> shape(x)
(4L, 3L)
>>> M,N = shape(x)
>>> x.shape = 3,4
>>> x.shape
(3L, 4L)
>>> x.shape = 6,-1
>>> x.shape
(6L, 2L)
%-----------------------------------%
reshape

reshape transforms an array with one set of dimensions and to one with a different set, preserving the
number of elements. Arrays with dimensions M by N can be reshaped into an array with dimensions K
by L as long as M N = K L. The most useful call to reshape switches an array into a vector or vice versa.
>>> x = array([[1,2],[3,4]])
>>> y = reshape(x,(4,1))
>>> y
array([[1],
[2],
[3],
[4]])
>>> z=reshape(y,(1,4))
>>> z
array([[1, 2, 3, 4]])
>>> w = reshape(z,(2,2))
array([[1, 2],
[3, 4]])
The crucial implementation detail of reshape is that arrays are stored using row-major notation. Elements
in arrays are counted first across rows and then then down columns. reshape will place elements of the
old array into the same position in the new array and so after calling reshape, x (1) = y (1), x (2) = y (2),
and so on.
%-----------------------------------%
size
size returns the total number of elements in an array or matrix. size can be used as a function or an
attribute.
>>> x = randn(4,3)
>>> size(x)
12
>>> x.size
12
%-----------------------------------%
ndim
ndim returns the size of all dimensions or an array or matrix as a tuple. ndim can be used as a function or
an attribute .
>>> x = randn(4,3)
>>> ndim(x)
2
>>> x.ndim
2
%-----------------------------------%
tile
tile, along with reshape, are two of the most useful non-mathematical functions. tile replicates an array
according to a specified size vector. 


%-----------------------------------%
ravel
ravel returns a flattened view (1-dimensional) of an array or matrix. ravel does not copy the underlying
data (when possible), and so it is very fast.
>>> x = array([[1,2],[3,4]])
>>> x
array([[ 1, 2],
[ 3, 4]])
>>> x.ravel()
array([1, 2, 3, 4])
>>> x.T.ravel()
array([1, 3, 2, 4])
%-----------------------------------%
flatten
flatten works much like ravel, only that is copies the array when producing the flattened version.

%-----------------------------------%

