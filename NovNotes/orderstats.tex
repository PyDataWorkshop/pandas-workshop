Order statistics
 

numpy.amax() will find the max value in an array, and numpy.amin() does the same for the min value.

Examples
%--------------------------------------------------------------------------------------%
\begin{verbatim}

 

>>>>>> a = np.arange(4).reshape((2,2))
>>> a
array([[0, 1],
       [2, 3]])
>>> np.amin(a)           # Minimum of the flattened array
0
>>> np.amin(a, axis=0)   # Minima along the first axis
array([0, 1])
>>> np.amin(a, axis=1)   # Minima along the second axis
array([0, 2])
 

>>>>>> b = np.arange(5, dtype=np.float)
>>> b[2] = np.NaN
>>> np.amin(b)
nan
>>> np.nanmin(b)
0.0

\end{verbatim}
%--------------------------------------------------------------------------------------%


amin(a[, axis, out, keepdims])
 
Return the minimum of an array or minimum along an axis.
 


amax(a[, axis, out, keepdims])
 
Return the maximum of an array or maximum along an axis.
 


nanmin(a[, axis, out, keepdims])
 
Return minimum of an array or minimum along an axis, ignoring any NaNs.
 


nanmax(a[, axis, out, keepdims])
 
Return the maximum of an array or maximum along an axis, ignoring any
 


ptp(a[, axis, out])
 
Range of values (maximum - minimum) along an axis.
 


percentile(a, q[, axis, out, overwrite_input])
 
Compute the qth percentile of the data along the specified axis.
