% - https://dataaspirant.wordpress.com/2014/12/20/linear-regression-implementation-in-python/

ZooZoo have the following data set

% Table Here
 


No.    
 
        square_feet       
 
         price     
 


1
 
150
 
6450
 


2
 
200
 
7450
 


3
 
250
 
8450
 


4
 
300
 
9450
 


5
 
350
 
11450
 


6
 
400
 
15450
 


7
 
600
 
18450
 

 
 
 
 
 
 
About data set:
 ■Square feet is the  Area of house.
 ■Price is the corresponding cost of  that house.
 
Steps to Follow :
 ■As we learn linear regression we know that we have to find linear line for this data so that we can get  θ0 and θ1.
 ■As you remember our hypothesis equation looks like this
 


where:
 ■hθ(x) is nothing but the value price(which we are going to predicate ) for particular square_feet  ( means price is a linear function of square_feet)
 ■θ0 is a constant
 ■θ1 is the regression coefficient
 
As we clear what we have to do, let’s start coding.
 
STEP – 1 :
 ■First open your favorite text editor and name it as predict_house_price.py.
 ■The below packages we gonna use in our program ,so  copy them in your predict_house_price.py file.
 
  %=============================================================================%

\begin{framed}
\begin{verbatim}
# Required Packages

import matplotlib.pyplot as plt
import numpy as np
import pandas as pd
from sklearn import datasets, linear_model
\end{verbatim}


 ■Just run your code once. if your program is error free then most of the job was done. If you facing  any errors , this means you missed some  packages so please go to this packages page  
 ■Install all the packages in that blog post and run your code once again . This time most probably you will never face any problem.
 ■Means your program is error free now so we can go to STEP – 2.
  %=============================================================================%

STEP – 2
 ■I stored our data set in to a csv file with name input_data.csv
 ■So let’s write a function to get our data into X values ( square_feet  ) Y values (Price)
 
# Function to get data
def get_data(file_name):
 data = pd.read_csv(file_name)
 X_parameter = []
 Y_parameter = []
 for single_square_feet ,single_price_value in zip(data['square_feet'],data['price']):
       X_parameter.append([float(single_square_feet)])
       Y_parameter.append(float(single_price_value))
 return X_parameter,Y_parameter
 
Line 3:
 
Reading csv data to pandas DataFrame.
 
Line 6-9:
 
Converting pandas dataframe data to X_parameter and Y_parameter data returning them
 
So let’s print our X_parameters and Y_parameters
 
X,Y = get_data('input_data.csv')
print X
print Y
  %=============================================================================%

\subsection*{Script Output}

 [[150.0], [200.0], [250.0], [300.0], [350.0], [400.0], [600.0]]
[6450.0, 7450.0, 8450.0, 9450.0, 11450.0, 15450.0, 18450.0]
[Finished in 0.7s] 
Step – 3
 
we converted data to X\_parameters and Y\_parameter so let’s fit our X_parameters and Y_parameters to Linear Regression model
 
So we gonna write a function which will take  X\_parameters ,Y\_parameter and the value you gonna predict  as input and return the θ0 ,θ1  and predicted value
 
  %=============================================================================%
 
# Function for Fitting our data to Linear model
def linear_model_main(X_parameters,Y_parameters,predict_value):

 # Create linear regression object
 regr = linear_model.LinearRegression()
 regr.fit(X_parameters, Y_parameters)
 predict_outcome = regr.predict(predict_value)
 predictions = {}
 predictions['intercept'] = regr.intercept_
 predictions['coefficient'] = regr.coef_
 predictions['predicted_value'] = predict_outcome
 return predictions
 
Line 5-6:
 
First we are creating an linear model and the training it with our X_parameters and Y_parameters
 
Line 8-12:
 
we are creating one dictionary with name predictions and storing θ0 ,θ1  and predicted values. and returning predictions dictionary as an output.
 
So let’s call our function with predicting value as 700
 
X,Y = get_data('input_data.csv')
predictvalue = 700
result = linear_model_main(X,Y,predictvalue)
print "Intercept value " , result['intercept']
print "coefficient" , result['coefficient']
print "Predicted value: ",result['predicted_value']
 
 %=============================================================================%
 
Script Output:
 Intercept value 1771.80851064
coefficient [ 28.77659574]
Predicted value: [ 21915.42553191]
[Finished in 0.7s] 
Here Intercept value is nothing but   θ0 value and coefficient value is nothing but  θ1 value.
 
We got the predicted values as 21915.4255 means we done our job of predicting the house price.
 
For checking purpose we have to see how our data fit to linear regression.So we have to write a function which takes X_parameters and Y_parameters as input and show the linear line fitting for our data.
 
 
 
# Function to show the resutls of linear fit model
def show_linear_line(X_parameters,Y_parameters):
 # Create linear regression object
 regr = linear_model.LinearRegression()
 regr.fit(X_parameters, Y_parameters)
 plt.scatter(X_parameters,Y_parameters,color='blue')
 plt.plot(X_parameters,regr.predict(X_parameters),color='red',linewidth=4)
 plt.xticks(())
 plt.yticks(())
 plt.show()

 
So let call our show_linear_line Function
 
show_linear_line(X,Y)

