
\documentclass[11pt]{article} % use larger type; default would be 10pt

\usepackage[utf8]{inputenc} 
\usepackage{geometry} % to change the page dimensions
\geometry{a4paper} 
\usepackage{graphicx} 
\usepackage{booktabs} % for much better looking tables
\usepackage{array} % for better arrays (eg matrices) in maths
\usepackage{paralist} % very flexible & customisable lists (eg. enumerate/itemize, etc.)
\usepackage{verbatim} % adds environment for commenting out blocks of text & for better verbatim
\usepackage{subfig} 
\usepackage{framed}
\usepackage{subfiles}
\usepackage{fancyhdr} % This should be set AFTER setting up the page geometry
\pagestyle{fancy} % options: empty , plain , fancy
\renewcommand{\headrulewidth}{0pt} % customise the layout...
\lhead{}\chead{Data Analysis with Python}\rhead{}
\lfoot{}\cfoot{\thepage}\rfoot{}
%--------------------------------------------------------------------------------------------%
\usepackage{sectsty}
\allsectionsfont{\sffamily\mdseries\upshape} 
\usepackage[nottoc,notlof,notlot]{tocbibind} % Put the bibliography in the ToC
\usepackage[titles,subfigure]{tocloft} % Alter the style of the Table of Contents
\renewcommand{\cftsecfont}{\rmfamily\mdseries\upshape}
\renewcommand{\cftsecpagefont}{\rmfamily\mdseries\upshape} % No bold!
%--------------------------------------------------------------------------------------------%

\title{Brief Article}
\author{The Author}
%--------------------------------------------------------------------------------------------%

\begin{document}
%\maketitle
%--------------------------------------------------------------------------------------------%
\section{Creating Dates and Times}

Dates are created using date by providing integer values for year, month and day and times are created
using time using hours, minutes, seconds and microseconds.

\begin{framed}
\begin{verbatim}
>>> import datetime as dt

>>> yr, mo, dd = 2012, 12, 21

>>> dt.date(yr, mo, dd)
datetime.date(2012, 12, 21)

>>> hr, mm, ss, ms= 12, 21, 12, 21

>>> dt.time(hr, mm, ss, ms)
dt.time(12,21,12,21)

\end{verbatim}
\end{framed}

Dates created using date do not allow times, and dates which require a time stamp can be created using
datetime, which combine the inputs from date and time, in the same order.
\begin{framed}
\begin{verbatim}
>>> dt.datetime(yr, mo, dd, hr, mm, ss, ms)
datetime.datetime(2012, 12, 21, 12, 21, 12, 21)

\end{verbatim}
\end{framed}
%-----------------------------------------------------------------%


\begin{verbatim}
>>> datetime64(’2013’)
numpy.datetime64(’2013’)

>>> datetime64(’201309’)
numpy.datetime64(’201309’)

>>> datetime64(’20130901’)
numpy.datetime64(’20130901’)

>>> datetime64(’20130901T12:
00’) # Time
numpy.datetime64(’20130901T12:
00+0100’)

>>> datetime64(’20130901T12:
00:01’) # Seconds
numpy.datetime64(’20130901T12:
00:01+0100’)

>>> datetime64(’20130901T12:
00:01.123456789’) # Nanoseconds
numpy.datetime64(’20130901T12:
00:01.123456789+0100’)
\end{verbatim}

Date or time units can be explicitly included as the second input. The final example shows that rounding
can occur if the date input is not exactly representable using the date unit chosen.
%-----------------------------------------------------------------------%
\begin{framed}
\begin{verbatim}
>>> datetime64(’20130101T00’,’
h’)
numpy.datetime64(’20130101T00:
00+0000’,’h’)
>>> datetime64(’20130101T00’,’
s’)
numpy.datetime64(’20130101T00:
00:00+0000’)
>>> datetime64(’20130101T00’,’
ms’)
numpy.datetime64(’20130101T00:
00:00.000+0000’)
>>> datetime64(’20130101’,’
W’)
numpy.datetime64(’20121227’)
\end{verbatim}
\end{framed}

NumPy datetimes can also be initialized from arrays.
%-----------------------------------------------------------------------%
\begin{framed}
\begin{verbatim}
>>> dates = array([’20130901’,’
20130902’],
dtype=’datetime64’)
>>> dates
array([’20130901’,
’20130902’],
dtype=’datetime64[D]’)
>>> dates[0]
numpy.datetime64(’20130901’)
\end{verbatim}
\end{framed}
%-----------------------------------------------------------------------%
\end{document}

