

%----------------------------------------------------------------------------------------------------%
\documentclass[main.tex]{subfiles}
\begin{document}
\section{Data Structures}

pandas introduces two new data structures to Python - Series and DataFrame, both of which are built on top of NumPy (this means it's fast).

\begin{framed}
\begin{verbatim}
import pandas as pd
import numpy as np
import matplotlib.pyplot as plt
pd.set_option('max_columns', 50)
\end{verbatim}
\end{framed}
\subsection{Series}

A Series is a one-dimensional object similar to an array, list, or column in a table. It will assign a labeled index to each item in the Series. By default, each item will receive an index label from 0 to N, where N is the length of the Series minus one.

\begin{framed}
\begin{verbatim}
# create a Series with an arbitrary list
s = pd.Series([7, 'Heisenberg', 3.14, -1789710578, 'Happy Eating!'])
s
\end{verbatim}
\end{framed}
\begin{verbatim}
0                7
1       Heisenberg
2             3.14
3      -1789710578
4    Happy Eating!
dtype: object
\end{verbatim}
Alternatively, you can specify an index to use when creating the Series.

\begin{framed}
\begin{verbatim}
s = pd.Series([7, 'Heisenberg', 3.14, -1789710578, 'Happy Eating!'],
              index=['A', 'Z', 'C', 'Y', 'E'])
s
\end{verbatim}
\end{framed}
\begin{verbatim}
A                7
Z       Heisenberg
C             3.14
Y      -1789710578
E    Happy Eating!
dtype: object
\end{verbatim}
\end{document}
The Series constructor can convert a dictonary as well, using the keys of the dictionary as its index.

In [4]:
d = {'Chicago': 1000, 'New York': 1300, 'Portland': 900, 'San Francisco': 1100,
     'Austin': 450, 'Boston': None}
cities = pd.Series(d)
cities
Out[4]:
Austin            450
Boston            NaN
Chicago          1000
New York         1300
Portland          900
San Francisco    1100
dtype: float64
You can use the index to select specific items from the Series ...

In [5]:
cities['Chicago']
Out[5]:
1000.0
In [6]:
cities[['Chicago', 'Portland', 'San Francisco']]
Out[6]:
Chicago          1000
Portland          900
San Francisco    1100
dtype: float64
Or you can use boolean indexing for selection.

In [7]:
cities[cities < 1000]
Out[7]:
Austin      450
Portland    900
dtype: float64
That last one might be a little weird, so let's make it more clear - cities < 1000 returns a Series of True/False values, which we then pass to our Series cities, returning the corresponding True items.

In [8]:
less_than_1000 = cities < 1000
print less_than_1000
print '\n'
print cities[less_than_1000]
Austin            True
Boston           False
Chicago          False
New York         False
Portland          True
San Francisco    False
dtype: bool


Austin      450
Portland    900
dtype: float64

You can also change the values in a Series on the fly.

In [9]:
# changing based on the index
print 'Old value:', cities['Chicago']
cities['Chicago'] = 1400
print 'New value:', cities['Chicago']
Old value: 1000.0
New value: 1400.0

In [10]:
# changing values using boolean logic
print cities[cities < 1000]
print '\n'
cities[cities < 1000] = 750

print cities[cities < 1000]
Austin      450
Portland    900
dtype: float64


Austin      750
Portland    750
dtype: float64

What if you aren't sure whether an item is in the Series? You can check using idiomatic Python.

In [11]:
print 'Seattle' in cities
print 'San Francisco' in cities
False
True

Mathematical operations can be done using scalars and functions.

In [12]:
# divide city values by 3
cities / 3
Out[12]:
Austin           250.000000
Boston                  NaN
Chicago          466.666667
New York         433.333333
Portland         250.000000
San Francisco    366.666667
dtype: float64
In [13]:
# square city values
np.square(cities)
Out[13]:
Austin            562500
Boston               NaN
Chicago          1960000
New York         1690000
Portland          562500
San Francisco    1210000
dtype: float64
You can add two Series together, which returns a union of the two Series with the addition occurring on the shared index values. Values on either Series that did not have a shared index will produce a NULL/NaN (not a number).

In [14]:
print cities[['Chicago', 'New York', 'Portland']]
print'\n'
print cities[['Austin', 'New York']]
print'\n'
print cities[['Chicago', 'New York', 'Portland']] + cities[['Austin', 'New York']]
\end{verbatim}
\end{framed}
\begin{verbatim}
Chicago     1400
New York    1300
Portland     750
dtype: float64


Austin       750
New York    1300
dtype: float64


Austin       NaN
Chicago      NaN
New York    2600
Portland     NaN
dtype: float64
\end{verbatim}

Notice that because Austin, Chicago, and Portland were not found in both Series, they were returned with NULL/NaN values.

NULL checking can be performed with isnull and notnull.

In [15]:
# returns a boolean series indicating which values aren't NULL
cities.notnull()
Out[15]:
Austin            True
Boston           False
Chicago           True
New York          True
Portland          True
San Francisco     True
dtype: bool
In [16]:
# use boolean logic to grab the NULL cities
print cities.isnull()
print '\n'
print cities[cities.isnull()]
Austin           False
Boston            True
Chicago          False
New York         False
Portland         False
San Francisco    False
dtype: bool


Boston   NaN
dtype: float64

%---------------------------------------%
\newpage
\subsection{DataFrame}

% pandas - chapter 5 - DataFrame

A DataFrame is a tablular data structure comprised of rows and columns, akin to a spreadsheet, database table, or R's data.frame object. You can also think of a DataFrame as a group of Series objects that share an index (the column names).

For the rest of the tutorial, we'll be primarily working with DataFrames.

\subsection{Reading Data}

To create a DataFrame out of common Python data structures, we can pass a dictionary of lists to the DataFrame constructor.

Using the columns parameter allows us to tell the constructor how we'd like the columns ordered. By default, the DataFrame constructor will order the columns alphabetically (though this isn't the case when reading from a file - more on that next).

\begin{framed}
\begin{verbatim}
In [17]:
data = {'year': [2010, 2011, 2012, 2011, 2012, 2010, 2011, 2012],
        'team': ['Bears', 'Bears', 'Bears', 'Packers', 'Packers', 'Lions', 'Lions', 'Lions'],
        'wins': [11, 8, 10, 15, 11, 6, 10, 4],
        'losses': [5, 8, 6, 1, 5, 10, 6, 12]}
football = pd.DataFrame(data, columns=['year', 'team', 'wins', 'losses'])
print football
\end{verbatim}
\end{framed}
   year     team  wins  losses
0  2010    Bears    11       5
1  2011    Bears     8       8
2  2012    Bears    10       6
3  2011  Packers    15       1
4  2012  Packers    11       5
5  2010    Lions     6      10
6  2011    Lions    10       6
7  2012    Lions     4      12

Much more often, you'll have a dataset you want to read into a DataFrame. Let's go through several common ways of doing so.

%----------------%
\subsection{CSV}

Reading a CSV is as simple as calling the read_csv function. By default, the read_csv function expects the column separator to be a comma, but you can change that using the sep parameter.

In [18]:
%cd ~/Dropbox/tutorials/pandas/
/Users/greda/Dropbox/tutorials/pandas

In [19]:
# Source: baseball-reference.com/players/r/riverma01.shtml
!head -n 5 mariano-rivera.csv
Year,Age,Tm,Lg,W,L,W-L%,ERA,G,GS,GF,CG,SHO,SV,IP,H,R,ER,HR,BB,IBB,SO,HBP,BK,WP,BF,ERA+,WHIP,H/9,HR/9,BB/9,SO/9,SO/BB,Awards
1995,25,NYY,AL,5,3,.625,5.51,19,10,2,0,0,0,67.0,71,43,41,11,30,0,51,2,1,0,301,84,1.507,9.5,1.5,4.0,6.9,1.70,
1996,26,NYY,AL,8,3,.727,2.09,61,0,14,0,0,5,107.2,73,25,25,1,34,3,130,2,0,1,425,240,0.994,6.1,0.1,2.8,10.9,3.82,CYA-3MVP-12
1997,27,NYY,AL,6,4,.600,1.88,66,0,56,0,0,43,71.2,65,17,15,5,20,6,68,0,0,2,301,239,1.186,8.2,0.6,2.5,8.5,3.40,ASMVP-25
1998,28,NYY,AL,3,0,1.000,1.91,54,0,49,0,0,36,61.1,48,13,13,3,17,1,36,1,0,0,246,233,1.060,7.0,0.4,2.5,5.3,2.12,

In [20]:
from_csv = pd.read_csv('mariano-rivera.csv')
from_csv.head()
\end{verbatim}
\end{framed}
\begin{verbatim}

Year	Age	Tm	Lg	W	L	W-L%	ERA	G	GS	GF	CG	SHO	SV	IP	H	R	ER	HR	BB	IBB	SO	HBP	BK	WP	BF	ERA+	WHIP	H/9	HR/9	BB/9	SO/9	SO/BB	Awards
0	 1995	 25	 NYY	 AL	 5	 3	 0.625	 5.51	 19	 10	 2	 0	 0	 0	 67.0	 71	 43	 41	 11	 30	 0	 51	 2	 1	 0	 301	 84	 1.507	 9.5	 1.5	 4.0	 6.9	 1.70	 NaN
1	 1996	 26	 NYY	 AL	 8	 3	 0.727	 2.09	 61	 0	 14	 0	 0	 5	 107.2	 73	 25	 25	 1	 34	 3	 130	 2	 0	 1	 425	 240	 0.994	 6.1	 0.1	 2.8	 10.9	 3.82	 CYA-3MVP-12
2	 1997	 27	 NYY	 AL	 6	 4	 0.600	 1.88	 66	 0	 56	 0	 0	 43	 71.2	 65	 17	 15	 5	 20	 6	 68	 0	 0	 2	 301	 239	 1.186	 8.2	 0.6	 2.5	 8.5	 3.40	 ASMVP-25
3	 1998	 28	 NYY	 AL	 3	 0	 1.000	 1.91	 54	 0	 49	 0	 0	 36	 61.1	 48	 13	 13	 3	 17	 1	 36	 1	 0	 0	 246	 233	 1.060	 7.0	 0.4	 2.5	 5.3	 2.12	 NaN
4	 1999	 29	 NYY	 AL	 4	 3	 0.571	 1.83	 66	 0	 63	 0	 0	 45	 69.0	 43	 15	 14	 2	 18	 3	 52	 3	 1	 2	 268	 257	 0.884	 5.6	 0.3	 2.3	 6.8	 2.89	 ASCYA-3MVP-14
\end{verbatim}
Our file had headers, which the function inferred upon reading in the file. Had we wanted to be more explicit, we could have passed header=None to the function along with a list of column names to use:

In [21]:
# Source: pro-football-reference.com/players/M/MannPe00/touchdowns/passing/2012/
!head -n 5 peyton-passing-TDs-2012.csv
1,1,2012-09-09,DEN,,PIT,W 31-19,3,71,Demaryius Thomas,Trail 7-13,Lead 14-13*
2,1,2012-09-09,DEN,,PIT,W 31-19,4,1,Jacob Tamme,Trail 14-19,Lead 22-19*
3,2,2012-09-17,DEN,@,ATL,L 21-27,2,17,Demaryius Thomas,Trail 0-20,Trail 7-20
4,3,2012-09-23,DEN,,HOU,L 25-31,4,38,Brandon Stokley,Trail 11-31,Trail 18-31
5,3,2012-09-23,DEN,,HOU,L 25-31,4,6,Joel Dreessen,Trail 18-31,Trail 25-31

In [22]:
cols = ['num', 'game', 'date', 'team', 'home_away', 'opponent',
        'result', 'quarter', 'distance', 'receiver', 'score_before',
        'score_after']
no_headers = pd.read_csv('peyton-passing-TDs-2012.csv', sep=',', header=None,
                         names=cols)
no_headers.head()
Out[22]:
num	game	date	team	home_away	opponent	result	quarter	distance	receiver	score_before	score_after
0	 1	 1	 2012-09-09	 DEN	 NaN	 PIT	 W 31-19	 3	 71	 Demaryius Thomas	 Trail 7-13	 Lead 14-13*
1	 2	 1	 2012-09-09	 DEN	 NaN	 PIT	 W 31-19	 4	 1	 Jacob Tamme	 Trail 14-19	 Lead 22-19*
2	 3	 2	 2012-09-17	 DEN	 @	 ATL	 L 21-27	 2	 17	 Demaryius Thomas	 Trail 0-20	 Trail 7-20
3	 4	 3	 2012-09-23	 DEN	 NaN	 HOU	 L 25-31	 4	 38	 Brandon Stokley	 Trail 11-31	 Trail 18-31
4	 5	 3	 2012-09-23	 DEN	 NaN	 HOU	 L 25-31	 4	 6	 Joel Dreessen	 Trail 18-31	 Trail 25-31
pandas various reader functions have many parameters allowing you to do things like skipping lines of the file, parsing dates, or specifying how to handle NA/NULL datapoints.

There's also a set of writer functions for writing to a variety of formats (CSVs, HTML tables, JSON). They function exactly as you'd expect and are typically called \texttt{to\_format}:
\begin{framed}
\begin{verbatim}
my_dataframe.to_csv('path_to_file.csv')
\end{verbatim}
\end{framed}
Take a look at the IO documentation to familiarize yourself with file reading/writing functionality.
%--------------------------------------------------------------------------------------%
\subsection{Excel}

Know who hates VBA? Me. I bet you do, too. Thankfully, pandas allows you to read and write Excel files, so you can easily read from Excel, write your code in Python, and then write back out to Excel - no need for VBA.

Reading Excel files requires the xlrd library. You can install it via pip (\texttt{pip install xlrd}).

Let's first write a DataFrame to Excel.

\begin{framed}
\begin{verbatim}
# this is the DataFrame we created from a dictionary earlier
print football.head()
\end{verbatim}
\end{framed}
   year     team  wins  losses
0  2010    Bears    11       5
1  2011    Bears     8       8
2  2012    Bears    10       6
3  2011  Packers    15       1
4  2012  Packers    11       5

\begin{framed}
\begin{verbatim}
# since our index on the football DataFrame is meaningless, let's not write it
football.to_excel('football.xlsx', index=False)

!ls -l *.xlsx
-rw-r--r--  1 greda  staff  5618 Oct 26 00:44 football.xlsx

# delete the DataFrame
del football

# read from Excel
football = pd.read_excel('football.xlsx', 'sheet1')
print football
\end{verbatim}
\end{framed}
%----------------------------------------------------------------%
\begin{verbatim}
   year     team  wins  losses
0  2010    Bears    11       5
1  2011    Bears     8       8
2  2012    Bears    10       6
3  2011  Packers    15       1
4  2012  Packers    11       5
5  2010    Lions     6      10
6  2011    Lions    10       6
7  2012    Lions     4      12
\end{verbatim}


\subsection{Databases}

pandas also has some support for reading/writing DataFrames directly from/to a database [docs]. You'll typically just need to pass a connection object to the \texttt{read\_frame} or \texttt{write\_frame} functions within the pandas.io module.

Note that \texttt{write\_frame} executes as a series of INSERT INTO statements and thus trades speed for simplicity. If you're writing a large DataFrame to a database, it might be quicker to write the DataFrame to CSV and load that directly using the database's file import arguments.

\begin{framed}
\begin{verbatim}
from pandas.io import sql
import sqlite3

conn = sqlite3.connect('/Users/greda/Dropbox/gregreda.com/_code/towed')
query = "SELECT * FROM towed WHERE make = 'FORD';"

results = sql.read_frame(query, con=conn)
print results.head()
\end{verbatim}
\end{framed}
\begin{verbatim}
     tow_date  make style model color    plate state        towed_address  \
0  01/19/2013  FORD    LL         RED  N786361    IL  400 E. Lower Wacker   
1  01/19/2013  FORD    4D         GRN  L307211    IL    701 N. Sacramento   
2  01/19/2013  FORD    4D         GRY  P576738    IL    701 N. Sacramento   
3  01/19/2013  FORD    LL         BLK  N155890    IL        10300 S. Doty   
4  01/19/2013  FORD    LL         TAN  H953638    IL        10300 S. Doty   

             phone inventory  
0   (312) 744-7550    877040  
1   (773) 265-7605   6738005  
2   (773) 265-7605   6738001  
3  (773) 568-8495    2699210  
4  (773) 568-8495    2699209  

\end{verbatim}

\end{document}
