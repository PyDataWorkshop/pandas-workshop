key1 key2
one a 1 2
b 3 4
c 5 6
d 7 8
two a 9 10
b 11 12
c 13 14
d 15 16
In some cases, a table might not have a fixed delimiter, using whitespace or some other
pattern to separate fields. In these cases, you can pass a regular expression as a delimiter
for read_table. Consider a text file that looks like this:
In [858]: list(open('ch06/ex3.txt'))
Out[858]:
[' A B C\n',
'aaa -0.264438 -1.026059 -0.619500\n',
'bbb 0.927272 0.302904 -0.032399\n',
'ccc -0.264273 -0.386314 -0.217601\n',
'ddd -0.871858 -0.348382 1.100491\n']
While you could do some munging by hand, in this case fields are separated by a variable
amount of whitespace. This can be expressed by the regular expression \s+, so we have
then:
In [859]: result = pd.read_table('ch06/ex3.txt', sep='\s+')
In [860]: result
Out[860]:
A B C
aaa -0.264438 -1.026059 -0.619500
bbb 0.927272 0.302904 -0.032399
ccc -0.264273 -0.386314 -0.217601
ddd -0.871858 -0.348382 1.100491
Because there was one fewer column name than the number of data rows, read_table
infers that the first column should be the DataFrame’s index in this special case.
The parser functions have many additional arguments to help you handle the wide
variety of exception file formats that occur (see Table 6-2). For example, you can skip
the first, third, and fourth rows of a file with skiprows:
In [861]: !cat ch06/ex4.csv
# hey!
a,b,c,d,message
# just wanted to make things more difficult for you
# who reads CSV files with computers, anyway?
1,2,3,4,hello
5,6,7,8,world
9,10,11,12,foo
In [862]: pd.read_csv('ch06/ex4.csv', skiprows=[0, 2, 3])
Out[862]:
a b c d message
