%--------------------------------------------------------------------------------------------%

\documentclass[11pt]{article} % use larger type; default would be 10pt

\usepackage[utf8]{inputenc} 
\usepackage{geometry} % to change the page dimensions
\geometry{a4paper} 
\usepackage{graphicx} 
\usepackage{booktabs} % for much better looking tables
\usepackage{array} % for better arrays (eg matrices) in maths
\usepackage{paralist} % very flexible & customisable lists (eg. enumerate/itemize, etc.)
\usepackage{verbatim} % adds environment for commenting out blocks of text & for better verbatim
\usepackage{subfig} 
\usepackage{subfiles}
\usepackage{framed}
\usepackage{subfiles}
\usepackage{fancyhdr} % This should be set AFTER setting up the page geometry
\pagestyle{fancy} % options: empty , plain , fancy
\renewcommand{\headrulewidth}{0pt} % customise the layout...
\lhead{}\chead{Data Analysis with Python}\rhead{}
\lfoot{}\cfoot{\thepage}\rfoot{}
%--------------------------------------------------------------------------------------------%
\usepackage{sectsty}
\allsectionsfont{\sffamily\mdseries\upshape} 
\usepackage[nottoc,notlof,notlot]{tocbibind} % Put the bibliography in the ToC
\usepackage[titles,subfigure]{tocloft} % Alter the style of the Table of Contents
\renewcommand{\cftsecfont}{\rmfamily\mdseries\upshape}
\renewcommand{\cftsecpagefont}{\rmfamily\mdseries\upshape} % No bold!
%--------------------------------------------------------------------------------------------%

\title{Brief Article}
\author{The Author}
%--------------------------------------------------------------------------------------------%

\begin{document}
\large
\setcounter{tocdepth}{2}
\tableofcontents
\newpage
%-----------------------------------------------------------------------------------%
\subsection*{An Introduction to Pandas}

%
%This tutorial will get you started with Pandas - a data analysis library for Python that is great for data preparation, joining, and ultimately generating well-formed, tabular data that's easy to use in a variety of visualization tools or (as we will see here) machine learning applications. 
%
%------------%
\textbf{Book Review}\\
\textbf{\textit{Python for Data Analysis}} is concerned with the nuts and bolts of manipulating, processing, cleaning, and crunching data in Python. It is also a practical, modern introduction to scientific computing in Python, tailored for data-intensive applications. This is a book about the parts of the Python language and libraries you’ll need to effectively solve a broad set of data analysis problems. This book is not an exposition on analytical methods using Python as the implementation language.

Written by Wes McKinney, the main author of the pandas library, this hands-on book is packed with practical cases studies. It’s ideal for analysts new to Python and for Python programmers new to scientific computing.

\begin{itemize}
\item Use the IPython interactive shell as your primary development environment
\item Learn basic and advanced NumPy (Numerical Python) features
\item Get started with data analysis tools in the pandas library
\item Use high-performance tools to load, clean, transform, merge, and reshape data
\item Create scatter plots and static or interactive visualizations with matplotlib
\item Apply the pandas groupby facility to slice, dice, and summarize datasets
\item Measure data by points in time, whether it’s specific instances, fixed periods, or intervals
\item Learn how to solve problems in web analytics, social sciences, finance, and economics, through detailed examples
\end{itemize}
\end{document}