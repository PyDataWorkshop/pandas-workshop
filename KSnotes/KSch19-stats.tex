%	19.1 Simulating Random Variables	
%		19.1.1 Core Random Number Generators
%		19.1.2 Random Array Functions
%		19.1.3 Select Random Number Generators
%	19.2 Simulation and Random Number Generation	
%		19.2.1 State
%		19.2.2 Seed
%		19.2.3 Replicating Simulation Data
%	19.3 Statistics Functions	
%	19.4 Continuous Random Variables	
%		19.4.1 Example: gamma
%		19.4.2 Important Distributions
%		19.4.3 Frozen Random Variable Object
%	19.5 Select Statistics Functions	
%	19.6 Select Statistical Tests	

\documentclass[KSmain.tex]{subfiles} 
\begin{document} 
\section{Probability and Statistics Functions}
\subsection{Random Number Generation with NumPy}
% 19.1 Simulating Random Variables
% - 19.1.1 Core Random Number Generators
NumPy random number generators are all stored in the module \texttt{numpy.random}. 
These can be imported with using \texttt{import numpy as np} and then calling \texttt{np.random.rand}, for example, or 
by importing \texttt{import numpy.random as rnd} and \texttt{using rnd.rand.1}.
\subsubsection{\texttt{rand}, \texttt{random\_sample}}
\texttt{rand} and \texttt{random\_sample }are uniform randomnumber generators whichare
 identicalexceptthat rand takes a variable number 
of integer inputs – one for each dimension – while \texttt{random\_sample} takes a n-element tuple. 

\texttt{random\_sample} is the preferred NumPy function, and \texttt{rand} is a convenience function primarily for \textit{MATLAB} users.
\begin{framed}
\begin{verbatim}
>>> x = rand(3,4,5) 
>>> y = random_sample((3,4,5))
\end{verbatim}
\end{framed}
%===============================================%

% -------19.1 Simulating Random Variables
% -------19.1.1 Core Random Number Generators

\subsubsection{\texttt{randn, standard\_normal}}
randn and standard\_normal are standard normal random number generators. randn, like rand, takes a
variable number of integer inputs, and standard\_normal takes an n-element tuple. Both can be called
with no arguments to generate a single standard normal (e.g. randn()). \texttt{standard\_normal }is the preferred
NumPy function, and randn is a convenience function primarily for MATLAB users .
\begin{framed}
\begin{verbatim}
>>> x = randn(3,4,5)
>>> y = standard_normal((3,4,5))
\end{verbatim}
\end{framed}
\subsubsection{\texttt{randint, random\_integers}}
\texttt{randint} and \texttt{random\_integers} are uniform integer random number generators which take 3 inputs, low,
high and size. Low is the lower bound of the integers generated, high is the upper and size is a n-element
tuple. \texttt{randint} and \texttt{random\_integers} differ in that \texttt{randint} generates integers exclusive of the value in high
(as do most Python functions), while \texttt{random\_integers} includes the value in high in its range.
\begin{framed}
\begin{verbatim}
>>> x = randint(0,10,(100))
>>> x.max() # Is 9 since range is [0,10)
9
>>> y = random_integers(0,10,(100))
>>> y.max() # Is 10 since range is [0,10]
10
\end{verbatim}
\end{framed}
% -------19.1.2 Random Array Functions
\subsubsection{\texttt{shuffle}}
shuffle randomly reorders the elements of an array in place.
\begin{framed}
\begin{verbatim}
>>> x = arange(10)
>>> shuffle(x)
>>> x
array([4, 6, 3, 7, 9, 0, 2, 1, 8, 5])
\end{verbatim}
\end{framed}
\subsubsection{\texttt{permutation}}
permutation returns randomly reordered elements of an array as a copy while not directly changing the
input.
\begin{framed}
\begin{verbatim}
>>> x = arange(10)
>>> permutation(x)
array([2, 5, 3, 0, 6, 1, 9, 8, 4, 7])
>>> x
array([0, 1, 2, 3, 4, 5, 6, 7, 8, 9])
\end{verbatim}
\end{framed}
% -------19.1.3 Select Random Number Generators
NumPy provides a large selection of random number generators for specific distribution. All take between
0 and 2 required inputs which are parameters of the distribution, plus a tuple containing the size of the
output. All random number generators are in the module numpy.random.
%

% 19.1.2 Random Array Functions
%======================================================================== %
% 19.1.3 Select Random Number Generators
%\subsubsection{Select Random Number Generators}
%% NumPyprovidesalargeselectionofrandomnumbergeneratorsforspecificdistribution. 
%
%All takebetween 0 and 2 required inputs which are parameters of the distribution, plus a tuple containing the size of the output. Allrandomnumbergeneratorsareinthemodule 
%numpy.random.

% \subsection{Bernoulli}
% ThereisnoBernoulligenerator. Instead usebinomial(1,p)to generateasingle draworbinomial(1,p,(10,10)) togenerate anarray where % p istheprobabilityofsuccess.

\newpage
%=================================%

%19.2 Simulation and Random Number Generation
\subsection{Simulation and Random Number Generation}
Computer simulated random numbers are usually constructed from very complex but ultimately deterministic
functions. Because they are not actually random, simulated random numbers are generally described
to as \textbf{pseudo-random}. 

All pseudo-random numbers inNumPy use one core random number generator
based on the \textbf{\textit{Mersenne Twister}}, a generator which can produce a very long series of pseudo-random
data before repeating (up to $2^19937 - 1$ non-repeating values)
.
\subsubsection{\texttt{RandomState}}
\texttt{RandomState} is the class used to control the random number generators. Multiple generators can be initialized
by RandomState.
\begin{framed}
	\begin{verbatim}
	>>> gen1 = np.random.RandomState()
	>>> gen2 = np.random.RandomState()
	>>> gen1.uniform() # Generate a uniform
	0.6767614077579269
	>>> state1 = gen1.get_state()
	>>> gen1.uniform()
	0.6046087317893271
	>>> gen2.uniform() # Different, since gen2 has different seed
	0.04519705909244154
	>>> gen2.set_state(state1)
	>>> gen2.uniform() # Same uniform as gen1 after assigning state
	0.6046087317893271
	\end{verbatim}
\end{framed}
%==============%
% 19.2.1
\subsubsection{\texttt{ State}}
Pseudo-random number generators track a set of values known as the \textit{state}. The state is usually a vector
which has the property that if two instances of the same pseudo-random number generator have the
same state, the sequence of pseudo-random numbers generated will be identical. The state of the default
random number generator can be read using \texttt{numpy.random.get\_state} and can be restored using
\texttt{numpy.random.set\_state}.
\begin{framed}
	\begin{verbatim}>>> st = get_state()
	>>> randn(4)
	array([ 0.37283499, 0.63661908, 1.51588209,
	1.36540624])
	>>> set_state(st)
	>>> randn(4)
	array([ 0.37283499, 0.63661908, 1.51588209,
	1.36540624])
	\end{verbatim}
\end{framed}
The two sequences are identical since they the state is the same when \texttt{randn} is called. The state is a 5-
element tuple where the second element is a 625 by 1 vector of unsigned 32-bit integers. In practice the
state should only be stored using get\_state and restored using \texttt{set\_state}.
\newpage

\subsubsection{\texttt{get\_state}}
%19.2
\texttt{get\_state()} gets the current state of the random number generator, which is a 5-element tuple. It can be
called as a function, in which case it gets the state of the default random number generator, or as a method
on a particular instance of RandomState.
\subsubsection*{\texttt{set\_state}}
%19.2.1
\texttt{set\_state(state)} sets the state of the random number generator. It can be called as a function, in which
case it sets the state of the default random number generator, or as a method on a particular instance of
RandomState. \texttt{set\_state} should generally only be called using a state tuple returned by get\_state.

\subsubsection*{\texttt{seed}}
% 19.2.2 
\texttt{numpy.random.seed} is a more useful function for initializing the random number generator, and can be
used in one of two ways. \texttt{seed()} will initialize (or reinitialize) the random number generator using some
actual random data provided by the operating system.

\texttt{seed( s )} takes a vector of values (can be scalar) to
initialize the random number generator at particular state. seed( s ) is particularly useful for producing
simulation studies which are reproducible. In the following example, calls to \texttt{seed()} produce different
random numbers, since these reinitialize using random data from the computer, while calls to \texttt{seed(0)}
produce the same (sequence) of random numbers.
\begin{framed}
\begin{verbatim}
>>> seed()
>>> randn()
array([ 0.62968838])
>>> seed()
>>> randn()
array([ 2.230155])
>>> seed(0)
>>> randn()
array([ 1.76405235])
>>> seed(0)
>>> randn()
array([ 1.76405235])
\end{verbatim}
\end{framed}
NumPy always calls \texttt{seed()}  when the first randomnumberis generated. As a result. calling \texttt{standard\_normal()}
across two “fresh” sessions will not produce the same random number.

%=========================================================%
%19.3 - Statistical Functions

% - Median
% - Standard Deviation
% - Variance
% - Correlation
% - Covariance
% - Histograms
% - Histogram Plots
%========================================================== %
% 19.3
\subsection{Statistics Functions}
\subsubsection*{\texttt{mean}}
\texttt{mean} computes the average of an array. An optional second argument provides the axis to use (default is
to use entire array). \texttt{mean} can be used either as a function or as a method on an array.
\begin{framed}
\begin{verbatim}
>>> x = arange(10.0)
>>> x.mean()
4.5
>>> mean(x)
4.5
>>> x= reshape(arange(20.0),(4,5))
>>> mean(x,0)
231
array([ 7.5, 8.5, 9.5, 10.5, 11.5])
>>> x.mean(1)
array([ 2., 7., 12., 17.])
\end{verbatim}
\end{framed}
\subsubsection*{\texttt{median}}
\texttt{median} computed the median value in an array. An optional second argument provides the axis to use
(default is to use entire array).
\begin{framed}
\begin{verbatim}
>>> x= randn(4,5)
>>> x
array([[0.74448693,
0.63673031,
0.40608815,
0.40529852, 0.93803737],
[ 0.77746525, 0.33487689, 0.78147524, 0.5050722
, 0.58048329],
[0.51451403,
0.79600763,
0.92590814, 0.53996231,
0.24834136],
[0.83610656,
0.29678017, 0.66112691,
0.10792584, 1.23180865]])
>>> median(x)
0.45558017286810903
>>> median(x, 0)
array([0.62950048,
0.16997507,
0.18769355, 0.19857318,
0.59318936])
\end{verbatim}
\end{framed}
Note that when an array or axis dimension contains an even number of elements (n), median returns the
average of the 2 inner elements.
%======================%
\subsubsection*{\texttt{std}}
\texttt{std} computes the standard deviation of an array. An optional second argument provides the axis to use
(default is to use entire array). std can be used either as a function or as a method on an array.
\subsubsection*{\texttt{var}}
\texttt{var} computes the variance of an array. An optional second argument provides the axis to


\subsubsection*{\texttt{corrcoef}}
\texttt{corrcoef(x)} computes the correlation between the rows of a 2-dimensional array x . \texttt{corrcoef(x, y)} computes
the correlation between two 1- dimensional vectors. An optional keyword argument rowvar can be
used to compute the correlation between the columns of the input – this is corrcoef(x, rowvar=False)
and \texttt{corrcoef(x.T)} are identical

\begin{framed}
\begin{verbatim}
>>> x= randn(3,4)
>>> corrcoef(x)
array([[ 1. , 0.36780596, 0.08159501],
[ 0.36780596, 1. , 0.66841624],
[ 0.08159501, 0.66841624, 1. ]])
>>> corrcoef(x[0],x[1])
array([[ 1. , 0.36780596],
[ 0.36780596, 1. ]])
>>> corrcoef(x, rowvar=False)
array([[ 1. , 0.98221501,
0.19209871,
0.81622298],
[0.98221501,
1. , 0.37294497, 0.91018215],
[0.19209871,
0.37294497, 1. , 0.72377239],
[0.81622298,
0.91018215, 0.72377239, 1. ]])
>>> corrcoef(x.T)
array([[ 1. , 0.98221501,
0.19209871,
0.81622298],
[0.98221501,
1. , 0.37294497, 0.91018215],
[0.19209871,
0.37294497, 1. , 0.72377239],
[0.81622298,
0.91018215, 0.72377239, 1. ]])
\end{verbatim}
\end{framed}
\subsubsection*{\texttt{cov}}
\texttt{cov(x)} computes the covariance of an array x . \texttt{cov(x,y)} computes the covariance between two 1-dimensional
vectors. An optional keyword argument rowvar can be used to compute the covariance between the
columns of the input – this is \texttt{cov(x, rowvar=False)} and \texttt{cov(x.T)} are identical.
\subsubsection*{\texttt{histogram}}
\texttt{histogram} can be used to compute the histogram (empirical frequency, using k bins) of a set of data. An
optional second argument provides the number of bins. If omitted, \texttt{k =10} bins are used. histogram returns
two outputs, the first with a k-element vector containing the number of observations in each bin, and the
second with the k + 1 endpoints of the k bins.
\begin{framed}
\begin{verbatim}
>>> x = randn(1000)
>>> count, binends = histogram(x)
>>> count
array([ 7, 27, 68, 158, 237, 218, 163, 79, 36, 7])
>>> binends
array([3.06828057,
2.46725067,
1.86622077,
1.26519086,
0.66416096,
0.06313105,
0.53789885, 1.13892875, 1.73995866, 2.34098856,
2.94201846])
>>> count, binends = histogram(x, 25)
\end{verbatim}
\end{framed}
\subsubsection*{\texttt{histogram2d}}
\texttt{histogram2d(x,y)} computes a 2-dimensional histogram for 1-dimensional vectors. An optional keyword
argument bins provides the number of bins to use. bins can contain either a single scalar integer or a
2-element list or array containing the number of bins to use in each dimension.
%=========================================================%
% Page 228
\newpage
\subsection{Distributions}
\subsubsection{\texttt{normal}}
\texttt{normal()} generates draws from a standard Normal (Gaussian). \texttt{normal(mu, sigma)} generates draws from
a Normal with mean $\mu$ and standard deviation $\sigma$. \texttt{normal(mu, sigma, (10,10))} generates a 10 by 10 array
of draws from a Normal with mean $\mu$ and standard deviation $\sigma$. 

\texttt{normal(mu, sigma)} is equivalent to \texttt{mu + sigma $\ast$ standard\_normal()}.
\subsubsection{\texttt{poisson}}
\texttt{poisson()} generates a draw from a Poisson distribution with $\lambda = 1$. \texttt{poisson(lambda)} generates a draw
from a Poisson distribution with expectation $\lambda$. poisson(lambda, (10,10)) generates a 10 by 10 array of
draws from a Poisson distribution with expectation $\lambda$.
\subsubsection{\texttt{standard\_t}}
\texttt{standard\_t(nu)} generates a draw from a Student’s t with shape parameter $\nu$. \texttt{standard\_t(nu, (10,10))}
generates a 10 by 10 array of draws from a Student’s t with shape parameter $\nu$.
\subsubsection{\texttt{uniform}}
\texttt{uniform()} generates a uniform random variable on (0, 1). uniform(low, high) generates a uniform on
(l , h). \texttt{uniform(low, high, (10,10))} generates a 10 by 10 array of uniforms on (l , h).
%=========================================================%
\newpage

%========================================================================= %
% 19.4 Continuous Random Variables

\subsection{Continuous Random Variables}

SciPy contains a large number of functions for working with continuous random variables. Each function
resides in its own class (e.g. norm for Normal or gamma for Gamma), and classes expose methods for random
number generation, computing the PDF, CDF and inverse CDF, fitting parameters using MLE, and
computing various moments. The methods are listed below, where dist is a generic placeholder for the
distribution name in SciPy. 
%---------------------------%
\begin{itemize} 
	\item \texttt{dist.rvs}\\
	Pseudo-randomnumbergeneration. Generically, rvs is called using dist.rvs(*args, loc=0, scale=1, size=size)
	where size is an n-element tuple containing the size of the array to be generated.
	\item \texttt{dist.pdf}\\
	Probability density function evaluation for an array of data (element-by-element). Generically, pdf is
	called using \texttt{dist.pdf(x, *args, loc=0, scale=1)} where x is an array that contains the values to use when
	evaluating PDF.
	\item \texttt{dist.logpdf}\\
	Log probability density function evaluation for an array of data (element-by-element). Generically, logpdf
	is called using dist.logpdf(x, *args, loc=0, scale=1) where x is an array that contains the values to use
	when evaluating log PDF.
	\item \texttt{dist.cdf}\\
	Cumulative distribution function evaluation for an array of data (element-by-element). Generically, cdf
	is called using d\texttt{ist.cdf(x, *args, loc=0, scale=1)} where x is an array that contains the values to use
	when evaluating CDF.
	\item \texttt{dist.ppf}\\
	Inverse CDF evaluation (also known as percent point function) for an array of values between 0 and 1.
	Generically, ppf is called using \texttt{dist.ppf(p, *args, loc=0, scale=1)} where p is an array with all elements
	between 0 and 1 that contains the values to use when evaluating inverse CDF.
	\item \texttt{dist.fit}\\
	Estimate shape, location, and scale parameters from data by maximum likelihood using an array of data.
	Generically, fit is called using \texttt{dist.fit(data, *args, floc=0, fscale=1)} where data is a data array used
	to estimate the parameters. floc forces the location to a particular value (e.g. floc=0). \texttt{fscale} similarly
	forces the scale to a particular value (e.g. \texttt{fscale=1}) . It is necessary to use floc and/or fscale when
	computing MLEs if the distribution does not have a location and/or scale. For example, the gamma distribution
	is defined using 2 parameters, often referred to as shape and scale. In order to useMLto estimate
	parameters from a gamma, floc=0 must be used.
	\item \texttt{dist.median}\\
	Returns the median of the distribution. Generically, median is called using dist.median(*args, loc=0, scale=1).
	\item \texttt{dist.mean}\\
	Returns the mean of the distribution. Generically, mean is called using dist.mean(*args, loc=0, scale=1).
	\item\texttt{ dist.moment}\\
	nth non-centralmomentevaluation of the distribution. Generically, moment is called using dist.moment(r, *args,
	loc=0, scale=1) where r is the order of the moment to compute.
	\item \texttt{dist.varr}\\
	Returns the variance of the distribution. Generically, var is called using dist.var(*args, loc=0, scale=1).
	\item \texttt{dist.std}\\
	Returns the standard deviation of the distribution. Generically, std is called using dist.std(*args, loc=0, scale=1).
\end{itemize}
%------------------------------------------------------------------%
%19.4.1 Example: gamma
\subsubsection{Example}
The gamma distribution is used as an example. The gamma distribution takes 1 shape parameter a (a is
the only element of *args), which is set to 2 in all examples.
\begin{framed}
	\begin{verbatim}
		
		
		>>> import scipy.stats as stats
		>>> gamma = stats.gamma
		>>> gamma.mean(2), gamma.median(2), gamma.std(2), gamma.var(2)
		(2.0, 1.6783469900166608, 1.4142135623730951, 2.0)
		>>> gamma.moment(2,2) gamma.
		moment(1,2)**2 # Variance
		>>> gamma.cdf(5, 2), gamma.pdf(5, 2)
		(0.95957231800548726, 0.033689734995427337)
		>>> gamma.ppf(.95957231800548726, 2)
		5.0000000000000018
		>>> log(gamma.pdf(5, 2)) gamma.
		logpdf(5, 2)
		0.0
		>>> gamma.rvs(2, size=(2,2))
		array([[ 1.83072394, 2.61422551],
		[ 1.31966169, 2.34600179]])
		>>> gamma.fit(gamma.rvs(2, size=(1000)), floc = 0) # a, 0, shape
		(2.209958533078413, 0, 0.89187262845460313)
	\end{verbatim}
\end{framed}

\subsection{More Statistical Functions}
%19.5
\subsubsection*{\texttt{mode}}
mode computes the mode of an array. An optional second argument provides the axis to use (default is to
use entire array). Returns two outputs: the first contains the values of the mode, the second contains the
number of occurrences.
\begin{framed}
	\begin{verbatim}
	>>> x=randint(1,11,1000)
	>>> stats.mode(x)
	(array([ 4.]), array([ 112.]))
	\end{verbatim}
\end{framed}
\subsubsection*{\texttt{moment}}
moment computed the rth central moment for an array. An optional second argument provides the axis to
use (default is to use entire array).
\begin{framed}
	\begin{verbatim}
	>>> x = randn(1000)
	>>> moment = stats.moment
	>>> moment(x,2) moment(
	x,1)**2
	0.94668836546169166
	>>> var(x)
	0.94668836546169166
	>>> x = randn(1000,2)
	>>> moment(x,2,0) # axis 0
	array([ 0.97029259, 1.03384203])
	\end{verbatim}
\end{framed}
\subsubsection*{\texttt{skew}}
skew computes the skewness of an array. An optional second argument provides the axis to use (default is
to use entire array).
\begin{framed}
	\begin{verbatim}
	>>> x = randn(1000)
	>>> skew = stats.skew
	>>> skew(x)
	0.027187705042705772
	>>> x = randn(1000,2)
	>>> skew(x,0)
	array([ 0.05790773, 0.00482564])
	\end{verbatim}
\end{framed}
%------------------------------------------------------------%
\subsubsection*{\texttt{kurtosis}}
kurtosis computes the excess kurtosis (actual kurtosis minus 3) of an array. An optional second argument
provides the axis to use (default is to use entire array). Setting the keyword argument fisher=False will
compute the actual kurtosis.
\begin{framed}
	\begin{verbatim}
	>>> x = randn(1000)
	>>> kurtosis = stats.kurtosis
	>>> kurtosis(x)
	0.2112381820194531
	>>> kurtosis(x, fisher=False)
	2.788761817980547
	>>> kurtosis(x, fisher=False) kurtosis(
	x) # Must be 3
	3.0
	>>> x = randn(1000,2)
	>>> kurtosis(x,0)
	array([0.13813704,
	0.08395426])
	\end{verbatim}
\end{framed}
\subsubsection*{\texttt{pearsonr}}
pearsonr computes the Pearson correlation between two 1-dimensional vectors. It also returns the 2-
tailed p-value for the null hypothesis that the correlation is 0.
\begin{framed}
	\begin{verbatim}
	>>> x = randn(10)
	>>> y = x + randn(10)
	>>> pearsonr = stats.pearsonr
	>>> corr, pval = pearsonr(x, y)
	>>> corr
	0.40806165708698366
	>>> pval
	0.24174029858660467
	\end{verbatim}
\end{framed}
%---------------------%
\subsubsection*{\texttt{spearmanr}}
spearmanr computes the Spearmancorrelation (rank correlation). It can be used with a single 2-dimensional
array input, or 2 1-dimensional arrays. Takes an optional keyword argument axis indicating whether to
treat columns (0) or rows (1) as variables. If the input array has more than 2 variables, returns the correlation
matrix. If the input array as 2 variables, returns only the correlation between the variables.
\begin{framed}
	\begin{verbatim}
	>>> x = randn(10,3)
	>>> spearmanr = stats.spearmanr
	>>> rho, pval = spearmanr(x)
	>>> rho
	array([[ 1. , 0.02087009,
	0.05867387],
	[0.02087009,
	1. , 0.21258926],
	[0.05867387,
	0.21258926, 1. ]])
	>>> pval
	array([[ 0. , 0.83671325, 0.56200781],
	[ 0.83671325, 0. , 0.03371181],
	[ 0.56200781, 0.03371181, 0. ]])
	>>> rho, pval = spearmanr(x[:,1],x[:,2])
	>>> corr
	0.020870087008700869
	>>> pval
	0.83671325461864643
	\end{verbatim}
\end{framed}
%---------------------------------------------------------%
\subsubsection*{\texttt{linregress}}
linregress estimates a linear regression between 2 1-dimensional arrays. It takes two inputs, the independent
variables (regressors) and the dependent variable (regressand). Models always include a constant.
\begin{framed}
	\begin{verbatim}
	>>> x = randn(10)
	>>> y = x + randn(10)
	>>> linregress = stats.linregress
	>>> slope, intercept, rvalue, pvalue, stderr = linregress(x,y)
	>>> slope
	1.6976690163576993
	>>> rsquare = rvalue**2
	>>> rsquare
	0.59144988449163494
	>>> x.shape = 10,1
	>>> y.shape = 10,1
	>>> z = hstack((x,y))
	>>> linregress(z) # Alternative form, [x y]
	(1.6976690163576993,
	0.79983724584931648,
	0.76905779008578734,
	0.0093169560056056751,
	0.4988520051409559)
	
	\end{verbatim}
\end{framed}
% 19.6 Select Statistical Tests
\subsection{Select Statistical Tests}
% normaltest
% kstest
% twosample KS test
% Shapiro Test

\subsubsection*{\texttt{normaltest}}
\texttt{normaltest} tests for normality in an array of data. An optional second argument provides the axis to use
(default is to use entire array). Returns the test statistic and the p-value of the test. This test is a small
sample modified version of the Jarque-Bera test statistic.
\subsubsection*{\texttt{kstest}}
\texttt{kstest} implements the Kolmogorov-Smirnov test. Requires two inputs, the data to use in the test and the
distribution, which can be a string or a frozen random variable object. If the distribution is provided as
a string, then any required shape parameters are passed in the third argument using a tuple containing
these parameters, in order.
\begin{verbatim}
>>> x = randn(100)
>>> kstest = stats.kstest
>>> stat, pval = kstest(x, ’norm’)
>>> stat
0.11526423481470172
>>> pval
0.12963296757465059
>>> ncdf = stats.norm().cdf # No () on cdf to get the function
>>> kstest(x, ncdf)
(0.11526423481470172, 0.12963296757465059)
>>> x = gamma.rvs(2, size = 100)
>>> kstest(x, ’gamma’, (2,)) # (2,) contains the shape parameter
(0.079237623453142447, 0.54096739528138205)
>>> gcdf = gamma(2).cdf
>>> kstest(x, gcdf)
(0.079237623453142447, 0.54096739528138205)
\end{verbatim}

\subsubsection{\texttt{ks\_2samp}}
\texttt{ks\_2samp} implements a 2-sample version of the Kolmogorov-Smirnov test. It is called \texttt{ks\_2samp(x,y)}
where both inputs are 1-dimensonal arrays, and returns the test statistic and p-value for the null that
the distribution of x is the same as that of y .
\subsubsection{\texttt{shapiro}}
\texttt{shapiro }implements the Shapiro-Wilk test for normality on a 1-dimensional array of data. It returns the
test statistic and p-value for the null of normality.

\newpage
\section{Statsmodels}
\texttt{Statsmodels} is a Python module that allows users to explore data, estimate statistical models, and perform statistical tests. 
An extensive list of descriptive statistics, statistical tests, plotting functions, and result statistics are available for different types of 
data and each estimator. Researchers across fields may find that statsmodels fully meets their needs for statistical computing and data analysis 
in Python. 

Features include:


\begin{itemize}

\item Linear regression models

\item Generalized linear models

\item Discrete choice models

\item Robust linear models

\item Many models and functions for time series analysis

\item Nonparametric estimators

\item A collection of datasets for examples

\item A wide range of statistical tests

\item Input-output tools for producing tables in a number of formats (Text, LaTex, HTML) and for reading Stata files into NumPy and Pandas.

\item Plotting functions

\item Extensive unit tests to ensure correctness of results

\item Many more models and extensions in development

\end{itemize}

\end{document}
