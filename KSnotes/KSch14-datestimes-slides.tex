\documentclass[KSmainSlides.tex]{subfiles} 
\begin{document} 

%=======================================================================================%
\begin{frame}[fragile]
%--------------------------------------------------------------------------------------------%
\frametitle{Creating Dates and Times}
\begin{itemize}
\item Dates are created using date by providing integer values for year, month and day and times are created
using time using hours, minutes, seconds and microseconds.

\item Date and time manipulation is provided by a built-in Python module datetime. This chapter assumes that
datetime has been imported using import datetime as dt.
\end{itemize}

\end{frame}
%=======================================================================================%
\begin{frame}[fragile]
\frametitle{Creating Dates and Times}
\begin{framed}
\begin{verbatim}
>>> import datetime as dt

>>> yr, mo, dd = 2012, 12, 21

>>> dt.date(yr, mo, dd)
datetime.date(2012, 12, 21)

>>> hr, mm, ss, ms= 12, 21, 12, 21

>>> dt.time(hr, mm, ss, ms)
dt.time(12,21,12,21)

\end{verbatim}
\end{framed}
	\end{frame}
	%=======================================================================================%
	\begin{frame}[fragile]
	\frametitle{Creating Dates and Times}
Dates created using date do not allow times, and dates which require a time stamp can be created using
datetime, which combine the inputs from date and time, in the same order.
\begin{framed}
\begin{verbatim}
>>> dt.datetime(yr, mo, dd, hr, mm, ss, ms)
datetime.datetime(2012, 12, 21, 12, 21, 12, 21)

\end{verbatim}
\end{framed}
	\end{frame}
	%=======================================================================================%
	\begin{frame}[fragile]
		\frametitle{Creating Dates and Times}
\begin{framed}
\begin{verbatim}
>>> datetime64(’2013’)
numpy.datetime64(’2013’)

>>> datetime64(’201309’)
numpy.datetime64(’201309’)

>>> datetime64(’20130901’)
numpy.datetime64(’20130901’)

>>> datetime64(’20130901T12:
00’) # Time
numpy.datetime64(’20130901T12:
00+0100’)
\end{verbatim}
\end{framed}
\end{frame}
%=======================================================================================%
\begin{frame}[fragile]
	\frametitle{Creating Dates and Times}
	\begin{framed}
		\begin{verbatim}
>>> datetime64(’20130901T12:
00:01’) # Seconds
numpy.datetime64(’20130901T12:
00:01+0100’)

>>> datetime64(’20130901T12:
00:01.123456789’) # Nanoseconds
numpy.datetime64(’20130901T12:
00:01.123456789+0100’)
\end{verbatim}
\end{framed}
\end{frame}
%=======================================================================================%
\begin{frame}[fragile]
\frametitle{Creating Dates and Times}
\begin{itemize}
\item Date or time units can be explicitly included as the second input. 
\item The final example shows that rounding
can occur if the date input is not exactly representable using the date unit chosen.
\end{itemize}


\end{frame}
%=======================================================================================%
\begin{frame}[fragile]
	\begin{framed}
		\begin{verbatim}
>>> datetime64(’20130101T00’,’
h’)
numpy.datetime64(’20130101T00:
00+0000’,’h’)
>>> datetime64(’20130101T00’,’
s’)
numpy.datetime64(’20130101T00:
00:00+0000’)
>>> datetime64(’20130101T00’,’
ms’)
numpy.datetime64(’20130101T00:
00:00.000+0000’)
>>> datetime64(’20130101’,’
W’)
numpy.datetime64(’20121227’)
\end{verbatim}
\end{framed}


\end{frame}
%=======================================================================================%
\begin{frame}[fragile]
NumPy datetimes can also be initialized from arrays.
	\begin{framed}
		\begin{verbatim}

>>> dates = array([’20130901’,’
20130902’],
dtype=’datetime64’)
>>> dates
array([’20130901’,
’20130902’],
dtype=’datetime64[D]’)
>>> dates[0]
numpy.datetime64(’20130901’)
\end{verbatim}
\end{framed}

\end{frame}
%=======================================================================================%
\begin{frame}[fragile]
% PAGE 142-143
\frametitle{14.2 Dates Mathematics}
\begin{itemize}
\item Date-times and dates (but not times, and only within the same type) can be subtracted to produce a
timedelta, which consists of three values, days, seconds and microseconds.
\item  Time deltas can also be added
to dates and times compute different dates – although date types will ignore any information in the time
delta hour or millisecond fields.
\end{itemize}

\end{frame}
%=======================================================================================%
\begin{frame}[fragile]
\frametitle{14.2 Dates Mathematics}
	\begin{framed}
		\begin{verbatim}
>>> d1 = dt.datetime(yr, mo, dd, hr, mm, ss, ms)
>>> d2 = dt.datetime(yr + 1, mo, dd, hr, mm, ss, ms)
>>> d2d1
datetime.timedelta(365)

\end{verbatim}
\end{framed}

\end{frame}
\end{document}
%=======================================================================================%
\begin{frame}[fragile]
\frametitle{14.2 Dates Mathematics}

%Date Unit Common Name Range Time Unit Common Name Range
%Y Year 9.2  1018 years h Hour 1.0  1015 years
%M Month 7.6  1017 years m Minute 1.7  1013 years
%W Week 1.7  1017 years s Second 2.9  1012 years
%D Day 2.5  1016 years ms Millisecond 2.9  109 years

\end{frame}
%=======================================================================================%
\begin{frame}[fragile]
\frametitle{14.2 Dates Mathematics}

%us Microsecond 2.9  106 years
%ns Nanosecond 292 years
%ps Picosecond 106 days
%fs Femtosecond 2.6 hours
%as Attosecond 9.2 seconds
%Table 14.1: NumPy datetime64 range. The absolute range is January 1, 1970 plus the range.

\end{frame}
%=======================================================================================%
\begin{frame}[fragile]
\frametitle{14.2 Dates Mathematics}
\begin{framed}
	\begin{verbatim}>>> d2 + dt.timedelta(30,0,0)
datetime.datetime(2014, 1, 20, 12, 21, 12, 20)
>>> dt.date(2012,12,21) + dt.timedelta(30,12,0)
datetime.date(2013, 1, 20)
\end{verbatim}
\end{framed}
If times stamps are important, date types can be promoted to datetime using combine and a time.
\begin{framed}
\begin{verbatim}
>>> d3 = dt.date(2012,12,21)
>>> dt.datetime.combine(d3, dt.time(0))
datetime.datetime(2012, 12, 21, 0, 0)
\end{verbatim}
\end{framed}
\end{frame}
%=======================================================================================%
\begin{frame}[fragile]
	\frametitle{14.2 Dates Mathematics}
Values in dates, times and datetimes can be modified using replace through keyword arguments.

\begin{framed}
\begin{verbatim}
>>> d3 = dt.datetime(2012,12,21,12,21,12,21)
>>> d3.replace(month=11,day=10,hour=9,minute=8,second=7,microsecond=6)
datetime.datetime(2012, 11, 10, 9, 8, 7, 6)
\end{verbatim}
\end{framed}
\end{frame}
%=======================================================================================%
\begin{frame}[fragile]
\frametitle{14.3 Numpy datetime64}
\begin{itemize}
\item Version 1.7.0 of NumPy introduces a NumPy native datetime type known as datetime64 (to distinguish it
from the usual datetime type). 
\item The NumPy datetime type is considered experimental and is not fully supported
in the scientific python stack at the time of writing these notes. 
\item This said, it is already widely used
and should see complete support in the near future. 
\item Additionally, the native NumPy data type is generally
better suited to data storage and analysis and extends the Python datetime with additional features such
as business day functionality.
\end{itemize}

\end{frame}
%=======================================================================================%
\begin{frame}[fragile]
	\frametitle{14.3 Numpy datetime64}
\begin{itemize}
\item NumPy contains both datetime (datetime64) and timedelta (timedelta64) objects. These differ from
the standard Python datetime since they always store the datetime or timedelta using a 64-bit integer plus
a date or time unit. 
\item The choice of the date/time unit affects both the resolution of the datetime as well
as the permissible range. The unit directly determines the resolution - using a date unit of a day (’D’)
limits to resolution to days.
\item Using a date unit of a week (’\texttt{W}’) will allow a minimum of 1 week difference.
\item Similarly, using a time unit of a second (’\texttt{s}’) will allow resolution up to the second (but not millisecond).
The set of date and time units, and their range are presented in Table 14.1.
\end{itemize}

\end{frame}
%=======================================================================================%
\begin{frame}[fragile]
\frametitle{14.3 Numpy datetime64}
\begin{itemize}
\item NumPy datetimes can be initialized using either human readable strings or using numeric values.
\item  The
string initialization is simple and datetimes can be initialized using year only, year and month, the complete
date or the complete date including a time (and optional timezone).
\item  The default time resolution is
nanoseconds ($10^{-9}$) and $T$ is used to separate the time from the date.
\end{itemize}

\end{frame}
%=======================================================================================%
\end{document}

