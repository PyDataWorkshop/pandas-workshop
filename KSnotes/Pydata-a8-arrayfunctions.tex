\documentclass[Pydata.tex]{subfiles} 
\begin{document} 

\section{Useful Array and Matrix Functions}
Many functions operate exclusively on array inputs, including functions which are mathematical in nature,
for example computing the eigenvalues and eigenvectors and functions for manipulating the elements
of an array.

\subsection*{Views}
Views are computationally efficient methods to produce objects of one type which behave as other objects
of another type without copying data. For example, an array x can always be converted to a matrix using
\texttt{matrix(x)}, which will copy the elements in x. View “fakes” the call to matrix and only inserts a thin layer
so that x viewed as a matrix behaves like a matrix.
\subsubsection{\texttt{view}}
\texttt{view }can be used to produce a representation of an array, matrix or recarray as another type without copying
the data. Using view is faster than copying data into a new class.
\begin{framed}
\begin{verbatim}
>>> x = arange(5)
>>> type(x)
numpy.ndarray
>>> x.view(matrix)
matrix([[0, 1, 2, 3, 4]])
>>> x.view(recarray)
rec.array([0, 1, 2, 3, 4])
\end{verbatim}
\end{framed}
\subsubsection*{\texttt{asmatrix}, \texttt{mat}}
asmatrix and mat can be used to view an array as a matrix. This view is useful since matrix views will use
matrix multiplication by default.
\begin{framed}
\begin{verbatim}
>>> x = array([[1,2],[3,4]])
>>> x * x # Elementbyelement
array([[ 1, 4],
[ 9, 16]])
>>> mat(x) * mat(x) # Matrix multiplication
matrix([[ 7, 10],
[15, 22]])
\end{verbatim}
\end{framed}
Both commands are equivalent to using view(matrix).
\subsubsection{\texttt{asarray}}
asarray work in a similar matter as asmatrix, only that the view produced is that of ndarray. Calling
asarray is equivalent to using view(ndarray)

% Section 8.2
%=========================================================%

\subsection{Shape Information and Transformation}

%-----------------------------------%
\subsubsection*{\texttt{shape}}

\texttt{shape} returns the size of all dimensions or an array or matrix as a tuple. shape can be called as a function
or an attribute. \texttt{shape} can also be used to reshape an array by entering a tuple of sizes. Additionally, the
new shape can contain -1 which indicates to expand along this dimension to satisfy the constraint that
the number of elements cannot change.
\begin{framed}
\begin{verbatim}
>>> x = randn(4,3)
>>> x.shape
(4L, 3L)
>>> shape(x)
(4L, 3L)
>>> M,N = shape(x)
>>> x.shape = 3,4
>>> x.shape
(3L, 4L)
>>> x.shape = 6,-1
>>> x.shape
(6L, 2L)
\end{verbatim}
\end{framed}
%-----------------------------------%
\subsubsection*{\texttt{reshape}}

\texttt{reshape}  transforms an array with one set of dimensions and to one with a different set, preserving the
number of elements. Arrays with dimensions M by N can be reshaped into an array with dimensions K
by L as long as $M \times N = K \times L$ (equal number of elements overall). 

\noindent The most useful call to reshape switches an array into a vector or vice versa.
\begin{framed}
\begin{verbatim}
>>> x = array([[1,2],[3,4]])
>>> y = reshape(x,(4,1))
>>> y
array([[1],
[2],
[3],
[4]])
>>> z=reshape(y,(1,4))
>>> z
array([[1, 2, 3, 4]])
>>> w = reshape(z,(2,2))
array([[1, 2],
[3, 4]])
\end{verbatim}
\end{framed}
The crucial implementation detail of reshape is that arrays are stored using row-major notation. Elements
in arrays are counted first across rows and then then down columns. reshape will place elements of the
old array into the same position in the new array and so after calling reshape, \texttt{x (1) = y (1), x (2) = y (2)},
and so on.

%-----------------------------------%
\subsubsection*{\texttt{ndim}}
\texttt{ndim} returns the size of all dimensions or an array or matrix as a tuple. \texttt{ndim} can be used as a function or
an attribute .
\begin{framed}
\begin{verbatim}
>>> x = randn(4,3)
>>> ndim(x)
2
>>> x.ndim
2
\end{verbatim}
\end{framed}
%-----------------------------------%
\subsubsection*{\texttt{size}}
\texttt{size} returns the total number of elements in an array or matrix. \texttt{size} can be used as a function or an
attribute.
\begin{framed}
\begin{verbatim}
>>> x = randn(4,3)
>>> size(x)
12
>>> x.size
12
\end{verbatim}
\end{framed}

%-----------------------------------%
\subsubsection*{\texttt{tile}}

\texttt{tile}, along with \texttt{reshape}, are two of the most useful non-mathematical functions. tile replicates an array
according to a specified size vector. To understand how tile functions, imagine forming an array composed
of blocks. The generic form of tile is \texttt{tile(X , (M, N ) )} where X is the array to be replicated, M is
the number of rows in the new block array, and N is the number of columns in the new block array. 
%
%For example, suppose X was an array
%MATRIX HERE
%and 
%MATRIX HERE
%was required. This could be accomplished by manually constructing y using hstack and vstack.
%an attribute .
\begin{framed}
\begin{verbatim}
>>> x = array([[1,2],[3,4]])
>>> z = hstack((x,x,x))
>>> y = vstack((z,z))
\end{verbatim}
\end{framed}
However, tile provides a much easier method to construct y
\begin{framed}
\begin{verbatim}
>>> w = tile(x,(2,3))
>>> y w
array([[0, 0, 0, 0, 0, 0],
[0, 0, 0, 0, 0, 0],
[0, 0, 0, 0, 0, 0],
[0, 0, 0, 0, 0, 0]])
\end{verbatim}
\end{framed}
\texttt{tile} has two clear advantages over manual allocation: 
\begin{itemize} 
\item First, tile can be executed using parameters determined
at run-time, such as the number of explanatory variables in a model,
\item second tile can be
used for arbitrary dimensions. Manual array construction becomes tedious and error prone with as few
as 3 rows and columns. 
\end{itemize}
\texttt{repeat} is a related function which copies data is a less useful manner.

%-----------------------------------%
\subsubsection*{\texttt{ravel}}
\texttt{ravel} returns a flattened view (1-dimensional) of an array or matrix. \texttt{ravel} does not copy the underlying
data (when possible), and so it is very fast.
\begin{framed}
\begin{verbatim}
>>> x = array([[1,2],[3,4]])
>>> x
array([[ 1, 2],
[ 3, 4]])
>>> x.ravel()
array([1, 2, 3, 4])
>>> x.T.ravel()
array([1, 3, 2, 4])
\end{verbatim}
\end{framed}
%-----------------------------------%
\subsubsection*{\texttt{flatten}}
\texttt{flatten} works much like \texttt{ravel}, only that is copies the array when producing the flattened version.

%-----------------------------------%
\subsubsection*{\texttt{flat}}
\texttt{flat} produces a \texttt{numpy.flatiter} object (\texttt{flat} iterator) which is an iterator over a flattened view of an array.
Because it is an iterator, it is especially fast and memory friendly. flat can be used as an iterator in a for
loop or with slicing notation.
\begin{framed}
\begin{verbatim}
>>> x = array([[1,2],[3,4]])
>>> x.flat
<numpy.flatiter at 0x6f569d0>
>>> x.flat[2]
3
>>> x.flat[1:4] = 1
>>> x
array([[ 1, 1],
[1,
1]])
\end{verbatim}
\end{framed}
%=========================================================%

% Page 90 KS
\subsubsection*{\texttt{split}, \texttt{vsplit}, \texttt{hsplit}}
\texttt{vsplit} and \texttt{hsplit} split arrays and matrices vertically and horizontally, respectively. Both can be used to
split an array into n equal parts or into arbitrary segments, depending on the second argument. If scalar,
the array is split into n equal sized parts. 

\noindent If a 1 dimensional array, the array is split using the elements of
the array as break points. For example, if the array was \texttt{[2,5,8]}, the array would be split into 4 pieces using
[:2] , [2:5], [5:8] and [8:]. Both \texttt{vsplit} and \texttt{hsplit} are special cases of split, which can split along an
arbitrary axis.
\begin{framed}
\begin{verbatim}
>>> x = reshape(arange(20),(4,5))
>>> y = vsplit(x,2)
>>> len(y)
2
>>> y[0]
array([[0, 1, 2, 3, 4],
[5, 6, 7, 8, 9]])
>>> y = hsplit(x,[1,3])
>>> len(y)
3
>>> y[0]
array([[ 0],
[ 5],
[10],
[15]])
>>> y[1]
array([[ 1, 2],
[ 6, 7],
[11, 12],
[16, 17]])
\end{verbatim}
\end{framed}
%-------------------------------%
\subsubsection*{\texttt{delete}}
\texttt{delete} removes values from an array, and is similar to splitting an array, and then concatenating the values
which are not deleted. The form of \texttt{delete} is \texttt{delete(x,rc, axis)} where rc are the row or column indices to
delete, and axis is the axis to use (0 or 1 for a 2-dimensional array). If axis is omitted, \texttt{delete} operated on
the flattened array.
\begin{framed}
\begin{verbatim}
>>> x = reshape(arange(20),(4,5))
>>> delete(x,1,0) # Same as x[[0,2,3]]
array([[ 0, 1, 2, 3, 4],
[10, 11, 12, 13, 14],
[15, 16, 17, 18, 19]])
>>> delete(x,[2,3],1) # Same as x[:,[0,1,4]]
array([[ 0, 1, 4],
[ 5, 6, 9],
[10, 11, 14],
[15, 16, 19]])
>>> delete(x,[2,3]) # Same as hstack((x.flat[:2],x.flat[4:]))
array([ 0, 1, 4, 5, 6, 7, 8, 9, 10, 11, 12, 13, 14, 15, 16, 17, 18,
19])
\end{verbatim}
\end{framed}
%-------------------------------%
\subsubsection*{\texttt{squeeze}}
\texttt{squeeze} removes \textit{singleton} dimensions from an array, and can be called as a function or a method.
\begin{framed}
\begin{verbatim}
>>> x = ones((5,1,5,1))
>>> shape(x)
(5L, 1L, 5L, 1L)
>>> y = x.squeeze()
>>> shape(y)
(5L, 5L)
>>> y = squeeze(x)
\end{verbatim}
\end{framed}
%-------------------------------%
\subsubsection*{\texttt{fliplr} and \texttt{flipud}}
\texttt{fliplr} and \texttt{flipud} flip arrays in a left-to-right and up-to-down directions, respectively. \texttt{flipud} reverses
the elements in a 1-dimensional array, and \texttt{flipud(x)} is identical to \texttt{x[::1]}.\\
\texttt{fliplr} cannot be used with 1-dimensional arrays.
\begin{framed}
\begin{verbatim}
>>> x = reshape(arange(4),(2,2))
>>> x
array([[0, 1],
[2, 3]])
>>> fliplr(x)
array([[1, 0],
[3, 2]])
>>> flipud(x)
array([[2, 3],
[0, 1]])
\end{verbatim}
\end{framed}
%-------------------------------%
\subsubsection*{\texttt{diag}}
The behavior of \texttt{diag} differs depending depending on the form of the input. 
\begin{itemize}
\item If the input is a square array, it
will return a column vector containing the elements of the diagonal. 
\item If the input is an vector, it will return
an array containing the elements of the vector along its diagonal. 
\end{itemize}

Consider the following example:
\begin{framed}
\begin{verbatim}
>>> x = array([[1,2],[3,4]])
>>> x
array([[1, 2],
[3, 4]])
>>> y = diag(x)
>>> y
array([1, 4])
>>> z = diag(y)
>>> z
array([[1, 0],
[0, 4]])
\end{verbatim}
\end{framed}
%-------------------------------%
\subsubsection*{\texttt{triu}, \texttt{tril}}
\texttt{triu} and \texttt{tril} produce upper and lower triangular arrays, respectively.
\begin{framed}
\begin{verbatim}
>>> x = array([[1,2],[3,4]])
>>> triu(x)
array([[1, 2],
[0, 4]])
>>> tril(x)
array([[1, 0],
[3, 4]])

\end{verbatim}
\end{framed}

%===============================================================================%
\newpage
% 8.3 
\subsection{Some Useful Linear Algebra Functions}
%-------------------------------%
\subsubsection*{\texttt{det}}
\texttt{det} computes the determinant of a square matrix or array.
\begin{framed}
\begin{verbatim}
>>> x = matrix([[1,.5],[.5,1]])
>>> det(x)
0.75
\end{verbatim}
\end{framed}

%-------------------------------%
\subsubsection*{\texttt{solve}}

\texttt{solve} solves the system \textit{\textbf{Ax=b}} when X is square and invertible so that the solution is exact.
\begin{framed}
\begin{verbatim}
>>> A = array([[1.0,2.0,3.0],[3.0,3.0,4.0],[1.0,1.0,4.0]])
>>> b = array([[1.0],[2.0],[3.0]])
>>> solve(A,b)
array([[ 0.625],
[1.125],
[ 0.875]])
\end{verbatim}
\end{framed}

%-------------------------------%
\subsubsection*{\texttt{eig}}
\texttt{eig} computes the eigenvalues and eigenvectors of a square matrix. When used with one output, the eigenvalues
and eigenvectors are returned as a tuple.
\begin{framed}
\begin{verbatim}
>>> x = matrix([[1,.5],[.5,1]])
>>> val,vec = eig(x)
>>> vec*diag(val)*vec.T
matrix([[ 1. , 0.5],
[ 0.5, 1. ]])
\end{verbatim}
\end{framed}
\texttt{eigvals} can be used if only eigenvalues are needed.
\end{document}
