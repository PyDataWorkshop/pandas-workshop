\documentclass[KSmain.tex]{subfiles} 
\begin{document} 
% KS Chapter 22
\newpage
\section{File System Operations}
%---------------------------------------------------%
Manipulating files and directories is surprising useful when undertaking complex projects. The most important
file system commands are located in the modules os and shutil. This chapter assumes that
\begin{framed}
\begin{verbatim}
import os
import shutil
\end{verbatim}
\end{framed}
have been included.
%---------------------------------------------------%
\subsection{22.1 Changing the Working Directory}
The working directory is where files can be created and accessed without any path information. os.getcwd()
can be used to determine the current working directory, and os.chdir(path) can be used to change the
working directory, where path is a directory, such as /temp or c:\\temp.1 Alternatively, path can can be .. to
more up the directory tree.
\begin{framed}
\begin{verbatim}
pwd = os.getcwd()
os.chdir(’c:\\temp’)
os.chdir(r’c:\temp’) # Raw string, no need to escape \
os.chdir(’c:/temp’) # Identical
os.chdir(’..’) # Walk up the directory tree
os.getcwd() # Now in ’c:\\’
\end{verbatim}
\end{framed}
%---------------------------------------------------%
\subsection{22.2 Creating and Deleting Directories}
Directories can be created using os.mkdir(dirname), although it must be the case that the higher level directories
exist (e.g. to create /home/username/Python/temp, it /home/username/Python already exists). os.makedirs(dirname)
works similar to os.mkdir(dirname), except that is will create any higher level directories needed to create
the target directory.
%---------------------------------------------------%
Empty directories can be deleted using os.rmdir(dirname) – if the directory is not empty, an error
occurs. shutil.rmtree(dirname) works similarly to os.rmdir(dirname), except that it will delete the directory,
and any files or other directories contained in the directory.
\begin{framed}
\begin{verbatim}
os.mkdir(’c:\\temp\\test’)
os.makedirs(’c:/temp/test/level2/level3’) # mkdir will fail
os.rmdir(’c:\\temp\\test\\level2\\level3’)
shutil.rmtree(’c:\\temp\\test’) # rmdir fails, since not empty
\end{verbatim}
\end{framed}
%---------------------------------------------------%
\subsection{22.3 Listing the Contents of a Directory}
The contents of a directory can be retrieved in a list using os.listdir(dirname), or simply os.listdir(’.’)
to list the current working directory. The list returned contains all files and directories. os.path.isdir(
name ) can be used to determine whether a value in the list is a directory, and os.path.isfile(name)
can be used to determine if it is a file. os.path contains other useful functions for working with directory
listings and file attributes.
\begin{framed}
\begin{verbatim}
os.chdir(’c:\\temp’)
files = os.listdir(’.’)
for f in files:
if os.path.isdir(f):
print(f, ’ is a directory.’)
elif os.path.isfile(f):
print(f, ’ is a file.’)
else:
print(f, ’ is a something else.’)
\end{verbatim}
\end{framed}

A more sophisticated listing which accepts wildcards and is similar to dir (Windows) and ls (Linux)
can be constructed using the glob module.
\begin{framed}
\begin{verbatim}
import glob
files = glob.glob(’c:\\temp\\*.txt’)
for file in files:
print(file)
\end{verbatim}
\end{framed}
%---------------------------------------------------%
\subsection{22.4 Copying, Moving and Deleting Files}
File contents can be copied using shutil.copy( src , dest ), shutil.copy2( src , dest ) or shutil.copyfile(
src , dest ). These functions are all similar, and the differences are:
\begin{itemize}
\item shutil.copy will accept either a filename or a directory as dest. If a directory is given, the a file is
created in the directory with the same name as the original file
\item  shutil.copyfile requires a filename for dest.
\item  shutil.copy2 is identical to shutil.copy except that metadata, such as last access times, is also
copied.
\end{itemize}
Finally, shutil.copytree( src , dest ) will copy an entire directory tree, starting from the directory src to
the directory dest, which must not exist. shutil.move( src,dest) is similar to shutil.copytree, except that
it moves a file or directory tree to a new location. If preserving file metadata (such as permissions or file

\end{document}
